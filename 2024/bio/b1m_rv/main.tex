\documentclass{article}
\author{}
\date{\today}
\title{Estudo de Biologia}

\usepackage{preamble}

\begin{document}
\maketitle
\tableofcontents

\section{Conceitos Básicos de Ecologia}
\paragraph{Biosfera} Engloba todas as regiões do planeta Terra onde há seres vivos.
\paragraph{População biológica} É um conjunto de indivíduos da mesma espécie. Tipo, no Ítaca, exite uma população de seres humanos, assim como existe uma de mosquitos.
\paragraph{Comunidade biológica} É um conjunto de populações que vivem no mesmo lugar.
\paragraph{Biótopo} É o ambiente, constituído por partes abióticas (\textit{a} \textrightarrow{} não, \textit{bióticas} \textrightarrow{} orgânicas, logo, as partes não orgânicas).  Por exemplo, o biótopo de uma determinada comunidade aquática envolve a composição química e a temperatura da água.
\paragraph{Ecossistema} É a unidade formada por uma comunidade que forma um sistema estável com um biótopo.
\paragraph{Hábitat} É o ambiente onde vivem determinadas espécies ou comunidades. Se falarmos que macacos vivem na copa das árvores, significa que o hábitat daqueles macacos é a copa das árvores 
\paragraph{Nicho ecológico} É a função ecológica de uma espécie, o que representa a sua relação com o ambiente.

\section{Cadeias e teias alimentares, níveis tróficos}
Uma \textbf{teia alimentar} é uma representação das relações alimentares de um ecossistema, ou seja, de quem come quem e para onde a energia vai. Uma \textbf{cadeia alimentar} é um subconjunto de uma teia alimentar que é linear, iniciando nos produtores (seres autotróficos) e terminando no último consumidor 

\section{Ciclos Biogeoquímicos}
\subsection{Nitrogênio}
O nitrogênio é armazenado na forma de \ch{N2}, o gás nitrogênio na atmosfera. Seu ciclo envolve alguns processos:

\subsubsection{Fixação} 
Fórmula: \ch{N2 -> NH3}

Esse processo transforma o gás nitrogênio da atmosfera em amônia (\ch{NH3}) e pode acontecer por dois caminhos:

\begin{itemize}
    \item{
        Em plantas leguminosas, onde uma bactéria fixadora de nitrogênio (\textit{Rhizobium sp.}) se associa a suas raízes em uma relação simbiótica, fazendo um nódulo. Os rizóbios ganham alimento a partir das substâncias orgânicas produzidas pela planta, e a planta ganha o nitrogênio. Quando o rizóbio morre, o \ch{NH3} armazenado em seu corpo é liberado no solo
    }

    \item {
        Através de bactérias diazotróficas, leveduras ou fungos que fixem o nitrogênio diretamente, e o liberam imediatamente no solo.
    }
\end{itemize}

\subsubsection{Amonificação} 
Fórmula: \ch{NH3 -> NH4+ + OH4-}

\ch{NH3} é altamente tóxica para os seres vivos e insolúvel em água. Logo, é necessário transformá-la no íon amônio (\ch{NH4+}), processo que tem como sub-produto a hidroxila (\ch{OH4-});

\subsubsection{Nitrificação}
O processo de nitrificação transforma \ch{NH4+} em nitrito (\ch{NO2-}) e depois em nitrato (\ch{NH3-}). Ele se divide em duas partes:
\paragraph{Nitrosação} \ch{NH4+ -> NO2-}
Transforma amônio em nitrito, processo realizado pelas \textit{Nitrossomonas sp.}.
\paragraph{Nitratação} \ch{NO2- -> NO3-}
Transforma \ch{NO2-} em nitrato, processo realizado pelas \textit{Nitrobacter sp.}

\subsubsection{Desnitrificação} 
Fórmula: \ch{NO3- -> N2}
Volta ao início do processo, transformando nitrato em gás nitrogênio. Esse processo é realizado pelas \textit{Pseudomonas sp.}

\subsubsection{Outros processos}
A matéria orgânica também contém nitrogênio. Assim, microorganismos decompositores podem transformar excrementos e corpos mortos em \ch{NH3}. Plantas também podem assimilar o nitrogênio do solo na forma de \ch{NO3-}, ou, mais raramente, na forma de \ch{NH3}

\subsubsection{Adubação verde}
Adubação verde é o processo de plantar plantas leguminosas para tornar o solo mais rico em nitrogênio, já que ele é essencial para a produtividade das culturas. Pode ser realizado de duas formas: com o plantio sendo feito junto das plantas não leguminosas, a plantação consorciada, ou alternado com elas, na rotação de culturas.

\subsection{Fósforo}
O ciclo do fósforo é facinho (vai tomar no cu corretor), uma brisa de verão, porque não existe fósforo atmosférico. E, para facilitar ainda mais, ele só tem uma fórmula química: \ch{{PO4}^{3-}} o glorioso íon fosfato.

A ideia é que existem dois ciclos no do fósforo:

\paragraph{Biológico} É aquele onde existe fósforo nos seres vivos, que podem comer outros seres vivos e logo assimilar fósforo, e que também podem fazer excrementos com fósforo ou morrerem, fazendo com que sua matéria orgânica seja decomposta. Essa decomposição gera fósforo no solo, que pode ser assimilado pelas plantas, que podem ser comidas e lá vai bolinha.

\paragraph{Geológico} Acontece quando parte do fósforo sedimenta-se nas rochas. Pedra = geológico, e demora muito tempo para entrar e sair.

\subsection{Carbono}
Eu tô sem paciência para resumir isso, mas o princípio é que você tem \ch{CO2} na atmosfera, que abarca o processo de respiração celular (oxigênio \textrightarrow \ch{CO2}) e de fotossíntese, que assimila esse \ch{CO2} nas plantas e emite oxigênio. Isso já é um ciclo. O \ch{CO2} assimilado nas plantas pode ser comido por herbívoros, e essa matéria orgânica depois vai ser decomposta, fazendo com que o carbono volte para a atmosfera.

\subsection{Água}
Tem água nas nuvens \textrightarrow chove. Parte da chuva percola no chão e escoa. Daí tem vários processos para a água voltar para a atmosfera, como transpiração, evaporação, ou evapotranspiração (que nada mais é do que evaporação + transpiração).
\section{Sobre notação biológica}
\textit{Rhizobium sp.} expande para ``\textit{Rhizobium species}'', ou seja, todas as espécies de \textit{Rhizobium}.
\end{document}