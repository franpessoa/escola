\documentclass[12pt, titlepage]{article}
\usepackage{preamble}
\addbibresource{referencias.bib}

\author{Francisco Pessoa}
\affil{Colégio Ítaca}
\date{São Paulo, \today}
\title{Experimentos sobre o crescimento de bolores}

\begin{document}
\maketitle

\section*{Resumo}
Esse experimento buscou entender o surgimento e o crescimento de colônias de fungos. Esse crescimento foi feito em um caldo de água com amido de milho fervido, dividido em quatro placas de Petri, para que fossem analisadas diferentes possibilidades de surgimento e crescimento dos fungos. Uma das placas foi tampada imediatamente, outra foi resfriada e então tampada, uma foi deixada aberta em um saco plástico fechado com furos e outra foi deixada aberta e exposta ao ambiente. [resumo de conclusões vem aqui]

\section{Introdução teórica e objetivo}
Durante a história da biologia, duas principais explicações para o surgimento da vida foram formuladas: a abiogênese, hipótese que previa o surgimento de vida a partir de geração espontânea, e a biogênese, hipótese que previa que a vida veio de algum tipo de formação biológica, proveniente de algum ser vivo pré-existente. Com o avanço da metodologia científica e da tecnologia, hoje em dia a abiogênese foi completamente desacreditada e obsoletada, mas fez parte do pensamento biológico durante quase dois séculos e foi parte importante da constituição filosófica e religiosa do pensamento moderno \cite{levine-evers}.

\printbibliography

\end{document}

