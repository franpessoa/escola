\documentclass[12pt, titlepage]{article}
\usepackage{preamble}
\addbibresource{referencias.bib}

\author{Francisco Pessoa}
\affil{Colégio Ítaca}
\date{São Paulo, \today}
\title{Experimentos sobre o crescimento de bolores}

\begin{document}
\maketitle

\section*{Resumo}
Esse experimento buscou entender o surgimento e o crescimento de colônias de fungos. Esse crescimento foi feito em um caldo de água com amido de milho fervido, dividido em quatro placas de Petri, para que fossem analisadas diferentes possibilidades de surgimento e crescimento dos fungos. Uma das placas foi tampada imediatamente, outra foi resfriada e então tampada, uma foi deixada aberta em um saco plástico fechado com furos e outra foi deixada aberta e exposta ao ambiente. [resumo de conclusões vem aqui]

\section{Introdução teórica e objetivo}
\subsection{Biogênese, geração espontânea e abiogênese}
Durante a história da biologia, duas principais explicações para o surgimento da vida foram formuladas: a geração espontânea, hipótese que, previsivelmente, defende a geração de organismos vivos a partir de matéria não-viva, e a biogênese, teoria atualmente mais aceita que prevê que toda a vida é proveniente de algum ser vivo pré-existente. Com o avanço da metodologia científica e da tecnologia, hoje em dia a geração espontânea foi completamente desacreditada e obsoletada. Contudo, sua discussão é ainda pertinente visto fez parte do pensamento biológico durante quase dois séculos e auxiliou substancialmente a constituição do pensamento e método experimental científico, além de ter influenciado na constituição religiosa de muitas sociedades passadas, como a Grécia antiga, sociedades islâmicas antigas e predominou no pensamento científico da idade Média.

Para os propósitos deste texto, os termos \textit{abiogênese} e \textit{geração espontânea} serão utilizados intercambiavelmente, muito embora a geração espontânea descreva o pensamento segundo o qual frequentemente, organismos vivos poderiam surgir por meio de um processo reproduzível que faria parte do ciclo de vida de determinadas espécies, enquanto abiogênese descreve um processo pelo qual um único organismo vivo surge espontaneamente. Enquanto a geração espontânea seja hoje em dia uma hipótese completamente descartada, a abiogênese é uma maneira de se referir ao evento que gerou a primeira forma viva da Terra \cite{vlaardingerbroek2010abiogenesis}.

\subsection{Fungos}



\printbibliography

\end{document}

