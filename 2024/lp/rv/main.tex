\documentclass{article}
\usepackage{preamble}
\author{Francisco Pessoa}
\date{\today}
\title{Resumo - LP e Literatura}
\makeindex

\begin{document}
\maketitle
\tableofcontents

\section{LP}

\subsection{Linguagem e língua}
Uma \textbf{linguagem} é uma maneira de se comunicar. Pode ser qualquer coisa, desenho, música, placa de trânsito, gritos sem sentido, etc. Ela pode ser verbal ou não verbal. Uma \textbf{língua} é um tipo específico de linguagem verbal que acontece de maneira organizada dentro de um determinado grupo social. Todas as línguas são linguagens, mas nem todas as linguagens são línguas.

\subsection{Teoria dos signos, de Saussure}
\subsubsection{Definição}
Um \textbf{signo} é a unidade básica de uma linguagem, usado para se referir a algo. Ele é composto por um \textbf{significante}, a sua forma material de representação e um \textbf{significado}, uma definição abstrata.

Por exemplo, considere a palavra ``capitalismo''. O significante é a palavra em si - \mbox{c-a-p-i-t-a-l-i-s-m-o} (sim, para escrever um significante linguístico tem que separar as letras desse jeito) - enquanto o significado é o conceito do sistema social de produção caracterizado por visar lucro, usar mão de obra assalariada, ter livre mercado e blá blá blá eu preciso estudar história.

``Capitalismo'' é um signo linguístico, ou seja, faz parte de uma língua e é um signo verbal. Uma placa com um desenho de um hominho branco descendo a escada e uma seta é um signo não-verbal.

\subsubsection{Arbitrariedade e caráter convencional do signo}
Um signo é, necessariamente, arbitrário. Porque o capitalismo se chama capitalismo, e não inferno? Porque uma porta não se chama \mbox{cmgnhaliru}?  O signo linguístico é, por mais que tente se aproximar do seu significado, uma imposição. A porta se chama porta, pois algum ser humano de milhares de anos atrás olhou para aquilo e imaginou que um som parecido com o nosso ``porta'' poderia ser usado para representá-la, e então a palavra passou por milhares de transformações linguísticas até chegar no que falamos hoje.

Sendo arbitrários, os signos linguísticos são uma convenção social, ou seja, é um acordo de todos os falantes de português que o significante p-o-r-t-a está relacionado com o significado de ``coisa que fica na entrada para impedir a passagem de entes indesejados''. Ninguém me entenderia, por exemplo, se eu começasse a trocar \mbox{p-o-r-t-a} por \mbox{j-i-h-á}.

\subsubsection{Linearidade}
Signos, quando colocados um depois do outro, criam um significado maior, fazendo com que você tenha que lê-los, ouvi-los ou fala-los em uma ordem linear. Por exemplo, ``crianças, vamos comer'' é bem diferente de ``vamos comer crianças''.

Assim, existe uma ordem para os signos, que, na fala, é a ordem temporal de fala, e, na escrita, é da esquerda para a direita, de cima para baixo, da direita para a esquerda, etc.

Mesmo quando você lê ou fala uma única palavra, a linearidade importa. Visto que uma palavra não é lida de inteiro, mas sim letra por letra, assim como não é falada de inteiro, mas sim fonema por fonema.

\subsection{Elementos da comunicação, de Jakobson}
O linguista russo R. Jakobson (pronuncia-se iá-kób-sôn) propôs o seguinte modelo:

% deu muito trabalho fazer isso
% não faço ideia de como funciona
\begin{tikzpicture}[node distance=10cm, auto]
    \node (e) at (0,0) [block]  {Emissor};
    \node (r) [block, right of=e]  {Receptor};
    \draw [<->, thick] (e) -- node [text width=2.5cm,midway,above=1.5em] {Mensagem\\Código\\Referente\\Canal} (r);
\end{tikzpicture}

Esse modelo descreve seis \textbf{componentes da linguagem}:
\begin{itemize}
    \item {\textbf{Emissor}: quem fala, quem produz a mensagem}
    \item {\textbf{Receptor}: quem recebe a mensagem}
    \item {\textbf{Mensagem}: o conteúdo transmitido}
    \item {\textbf{Código}: o conjunto de signos usados na mensagem}
    \item {\textbf{Referente}: o contexto;}
    \item {\textbf{Canal}: o meio pelo qual a transmição acontece}
\end{itemize}

\subsubsection{Exemplos}
\begin{enumerate}
    \item {
        \textbf{Um livro}
        \begin{itemize}
            \item {\textbf{Emissor}: o autor do livro}
            \item {\textbf{Receptor}: quem estiver lendo, o leitor}
            \item {\textbf{Mensagem}: a história do livro}
            \item {\textbf{Código}: a língua portuguesa, ou qualquer outra que esteja sendo utilizada}
            \item {\textbf{Referente}: vai variar, mas são informações que o leitor precisa saber para interpretar corretamente. Por exemplo, informações sobre o autor, data e contexto histórico.}
            \item {\textbf{Canal}: a edição}
        \end{itemize}
    }

    \item {
        \textbf{Uma música}
        \begin{itemize}
            \item {\textbf{Emissor}: a banda ou o artista}
            \item {\textbf{Receptor}: o ouvinte}
            \item {\textbf{Mensagem}: a música}
            \item {\textbf{Código}: a língua utilizada, e elementos musicais, como instrumentalização, tempo, harmonia, ritmo, entre outros}
            \item {\textbf{Referente}: quem é a banda? Foi feito quando? Em que álbum está?}
            \item {\textbf{Canal}: uma plataforma de \textit{streaming}, um CD, um vinil, um arquivo, entre outros}
        \end{itemize}
    }
\end{enumerate}

\subsubsection{Enunciação, enunciado e conhecimentos}
\textbf{NOTA:} isso não vai cair diretamente na prova, mas é importante para entender o referente no modelo de comunicação.

O enunciado é a mensagem, o conteúdo. A enunciação, além dele, engloba também o \textbf{conhecimento linguístico}, que é o saber acerca do funcionamento da língua e vocabulário, o \textbf{conhecimento enciclopédico}, que é o saber de outras coisas, como fatos históricos ou outro livros, que quem interpreta têm, e o \textbf{conhecimento interacional}, que é o contexto da interação, como expressões faciais, localização, entonação, entre outras coisas relacionadas à conversa. São esses conhecimentos que formam o \textbf{Referente} no modelo de Jakobson.

\subsubsection{Funções da linguagem, função emotiva e referencial}
Os textos são classificados conforme o enfoque que colocam em um dos componentes da comunicação explicados acima. Por exemplo, um texto com foco no receptor está na função apelativa. Esses enfoques não são únicos, já que um texto pode completar mais de um, porém, nesses casos, identifica-se a função principal.

Abaixo estão as duas funções que estudamos nesse bimestre:

\paragraph{Função emotiva\label{femotiva}}
A função emotiva coloca o foco no emissor, revelando sua opinião ou estado. Seus marcadores linguísticos são a 1\textsuperscript{a} pessoa do singular (eu), interjeições e exclamações. É mais usada em poemas e em textos que retratam intimamente o escritor, como autobiografias.


\paragraph{Função referencial ou função denotativa}
A função referencial coloca o foco no referente (contexto), revelando informações objetivas. Seu marcador linguístico é a 3\textsuperscript{a} pessoa do singular (ele/ela). É mais usada em textos, obviamente, objetivos e denotativos, como artigos científicos e reportagens.

\subsection{Referenciação\label{referenciação}}
A referenciação é o processo de usar um \textbf{pronome} para retomar ou projetar outra ideia (\textbf{referente}) dentro de uma frase. Por exemplo,

\blockquote{
    Minha mãe é legal. Ela me deu um copo de água.
}

Aí, ocorre referenciação com o uso do ``ela'', um pronome que remete ao conceito de ``minha mãe''.

Existem duas categorias de referenciação: a anafórica e catafórica. A anáfora ocorre quando o pronome vem depois do referente, como na frase acima, e a catafórica quando o pronome vem antes, como em ``Preciso lhe dizer uma coisa: saia deste lugar imediatamente''.

\subsection{Coesão e coerência}
Coesão e coerência são duas coisas distintas que um texto deve ter para fazer sentido. Coerência refere-se a um processo temático, de ideias, fazendo com que ter um texto coerente signifique que, por exemplo, o texto não mude de tema subitamente, ou caia em contradição.

Coesão refere-se, por outro, a um processo linguístico, onde o texto precisa ter uma estruturação interna para poder ser bem interpretado. A coesão auxilia na construção da coerência.

Existem alguns processos para a construção da coesão:

\paragraph{Referenciação}
Ver \S\ref{referenciação}
.
\paragraph{Elipse}
É o ato de esconder ou subentender parte do texto, como no provérbio ``casa de ferreiro, espeto de pau'', onde fica subentendido que ``[em] casa de ferreiro, [o] espeto [é] de pau''.

\paragraph{Uso de conjunções}
Conjunções são uma classe morfológica usada para ligar orações no mesmo período. Um bom uso de conjunções cria uma relação mais clara entre elas, facilitando o entendimento de texto.

\paragraph{Coesão lexical}
Coesão lexical é a retomada de um termo utilizando um sinônimo, hipônimo, hiperônimo ou expressão de significado equivalente. Um hipônimo é uma palavra de significado mais restrito do que o original (trabalhador\textrightarrow pedreiro), enquanto o hiperônimo é o uso de uma palavra mais generalizada (cachorro\textrightarrow animal). Sinônimo é uma palavra de mesmo significado (belo\textrightarrow bonito) e ``expressão de significado equivalente'' é a mesma coisa, porém com uma expressão maior (belo\textrightarrow pessoa dotada de beleza).

\section{Literatura}
\subsection{Teoria Clássica dos Gêneros [literários]}
A Teoria Clássica dos Gêneros, proveniente da Grécia Antiga, separa os gêneros literários em três:

\paragraph{Lírico}
É onde se manifesta a subjetividade, a expressão do ``eu'' - o \textbf{eu lirico} - presente nas músicas e poemas.

\paragraph{Épico}
É o gênero das narrativas, das epopeias, presente nas novelas, contos e romances. Narra os feitos épicos de um herói ou de um povo, como a Ilíada e a Odisseia, misturando a história da sociedade com um mito.

\paragraph{Dramático}
É o gênero do drama, do diálogo, da ação. Presente no teatro. Aqui, todo o desenrolar da história acontece nas falas dos personagens, sem narradores. Separa-se, classicamente, na tragédia e na comédia.
\subparagraph{Tragédia} São grandes histórias que envolvem reis e seres superiores, fornecendo um significado generalizado sobre coisas grandiosas. Incluem uma peripécia, mudança de estado, que geralmente é da felicidade para a tristeza. Para Aristóteles (se eu não me engano), Édipo Rei seria um modelo de tragédia.
\subparagraph{Comédias} Eram peças populares, inferiores, com apenas o propósito de entreter.

\subsubsection{Imperfeições}
Primariamente, nenhum gênero é puro, e todos sofrem influências dos outros. É difícil encontrar uma peça puramente dramática, assim como é difícil encontrar um romance sem diálogos. Assim, não é uma definição que separa de modo claro as peças, deixando linhas tênues e questionáveis entre eles.


\subsection{Trovadorismo}
O trovadorismo foi um movimento literário que aconteceu em Portugal do século XIV e XV, no contexto histórico da idade Média e da expansão comercial marítima das Grandes navegações. Nessa época, predominava o sistema feudal, e o acesso ao letramento era algo extremamente elitizado.

Dividia-se nas seguintes categorias
\subsubsection{Canções Líricas}
\paragraph{Canções de amor} 
Eram as mais rebuscadas e mais linguisticamente ricas, ligadas ao contexto da nobreza, narrando o amor de um homem de classe social baixa por uma mulher nobre. Esse amor, impossível por não existirem casamentos entre classes diferentes, logo não correspondido, evocava a figura da coita amorosa, o sofrimento por amor.

Além disso, existia o conceito de vassalagem amorosa, de um amor que levava à serventia, e do amor cortês.

\paragraph{Canções de Amigo}
Eram mais simples, usando estruturas paralelísticas (\S\ref{eparalelistica}). Narra o ponto de vista de uma mulher lamentando a ausência de seu marido, ligando-se ao período histórico das grandes navegações, onde muitos maridos viam-se obrigados a deixar seus lares para participar nas tais navegações.

\subsubsection{Canções satíricas}
Canções satíricas são aquelas que tem a finalidade de humilhar publicamente uma figura.
\paragraph{Canções de escárnio} 
Eram aquelas que, de maneira mais delicada, satirizavam a pessoa sem dizer o nome. Assim, contribuíam para a criação de alguns esteriótipos sociais, e criticavam as condutas louváveis das canções líricas.

\paragraph{Canções de maldizer} 
Eram aquelas que, de maneira extremamente indelicada, criticavam e humilhavam a pessoa mostrando nome e tudo, sem nenhuma resguarda.

\subsection{Estrutura paralelística}\label{eparalelistica}
É uma maneira de estruturar poemas que repete várias vezes uma mesma formação gramatical. É um formato extremamente simples.

Fim!
Boas provas e vai na fé!
\end{document}