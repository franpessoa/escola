\documentclass[12pt]{article}
\usepackage{preamble}
\addbibresource{ref.bib}
\author{André Tarrio, Bianca Bueno, Francisco Pessoa, Tomás Zamboni}
\date{\today}
\affil{1°EM — Turma A}
\title{O mito de Ulisses (ou a Odisseia)}

\begin{document}

\maketitle

\section{Introdução e sinopse}
A Odisseia é um dos mitos mais antigos que conhecemos, junto da Ilíada. Escritas por Homero, elas eram cantadas, como grandes poemas, que ao longo do tempo sofreram mudanças e reinterpretações. As histórias são contadas em sequência: enquanto a Ilíada narra episódios da guerra de Troia, a Odisseia conta sobre a volta do protagonista Ulisses, um dos heróis do confronto, em seu retorno para a casa: a ilha de Ítaca. 

Ulisses e sua tripulação se deparam com enormes desafios: vão a uma espécie de mundo paralelo, onde tudo é desconhecido e ameaçador. Se depara com grandes dilemas e riscos em diversos episódios, marcados por encontros com outros personagens e forças nesse mundo paralelo. São alguns destes o encontro com os lotófagos, Polifemo, com Éolo, com os lestrigões, com Circe, Calipso, Tirésias, as Sereias, com Cila e Caríbdis, e com os feácios, até que ao fim volta para Ítaca e batalha os pretendentes que se instalaram em sua casa.

\section{Análise}
Não é proveitoso olhar para Ulisses como o herói grego clássico — aquele líder que sempre certo, carrega a razão e a força, e é completamente intangível para qualquer um. Mas sim, olha-lo como um ser humano, imperfeito, alguém que arma e cai em armadilhas, alguém que tem que fazer decisões difíceis, e alguém que pode, eventualmente, se deixar seduzir ou ter seu ego inflado. Foi um bom homem, em alguns momentos, um péssimo homem em outros: foi heroico e foi comum.

Ulisses tem qualidades: é astuto, esperto e corajoso. Mas ainda é ingênuo. Pode-se interpretar que, durante a história, Ulisses aprende, transformando-se no seu "eu verdadeiro", em Ítaca, a partir de uma figura deslocada para longe de seu lugar natal. Neste sentido, no começo ele é imaturo —  quando, por exemplo, obriga sua família a passar por dificuldades durante sua grande volta para casa, e já no primeiro episódio após a saída de Troia, na ilha de Polifemo, deixa seu orgulho ultrapassar a razão, mudando seu destino pela história inteira. Mesmo no final, mesmo após chegar em casa, Ulisses tem de aprender a ser paciente para evitar as armadilhas colocadas pelos pretendentes.

Ulisses tem à sua disposição um recurso importante para poder amadurecer: um tempo e uma espacialidade fora do universo dos homens, em uma barreira entre o natural e o divino. Naquele lugar, vive aventuras fantásticas, mas que são metáforas para o crescimento não somente dele, mas de um ser humano.

E ele aprende. O acontecimento das sereias pode ser visto, de uma maneira, como um exemplo disso, já que Ulisses por meio da confiança em seus companheiros e nos conselhos de Circe, evita a atração do canto das sereias, canto esse que massageava seu ego assim como havia sido com a vitória sobre Polifemo. E as sereias são um obstáculo onde tem de reconhecer a sua humanidade e sua fraqueza: não pode simplesmente passar por lá, mas sim tem de estar amarrado a mercê de vários entes externos.

Ulisses tem um destino: voltar para casa. Um destino que está destinado a cumprir, mas não pode: tem de primeiro tornar-se verdadeiramente humano e sair deste meta-mundo fora da realidade onde está. Tem de cumprir essa tarefa fazendo escolhas difíceis, que envolvem deixar seu passado e história para trás e sair definitivamente da realidade humana. O mito de Ulisses reflete sobre glória e honra; a relação da sociedade com a vida e a memória daqueles que já foram. No Hades, encontra Aquiles, que lamenta sua vida de glória e diz preferir ser um camponês miserável vivo do que um morto em glória. Ulisses enfrenta esses dilemas, entre os leitos sobre-naturais de Circe e da ninfa Calipso no esquecimento e a volta para casa na forma de um herói grego na derrubada de Troia. E por anos se vê nessa posição até que, um pouco por vontade própria e um pouco por vontade divina, parte de volta para casa. 

Durante essa volta, tenta se afirmar para sua tripulação como líder. Tenta e falha, visto que ao final da história volta sozinho, com parte das mortes provocadas por lapsos de liderança, como quando a tripulação é chacinada por Zeus após irritar os deuses contra os conselhos de Ulisses na ilha do Sol. Note que, nesses segmentos, não é uma liderança ruim, mas sim uma liderança ausente. Ainda assim, essa imposição funciona algumas vezes: é ele que é destinado a ver as sereias, na posição do único da tripulação privilegiado o suficiente para ouvir seu canto. E, ao final das contas, Ulisses havia predeterminado a morte de todos os seus companheiros quando foi desmedidamente orgulhoso na ilha de Polifemo; por mais que seja uma história que tem a metáfora de explicar a condição humana, Ulisses ainda é um herói, único, tal que somente ele poderia ter vencido aqueles desafios.

Isso evidencia-se quando analisamos sua figura como um marido infiel, que traiu sua esposa com Circe e Calipso. Atribuir isso à ingenuidade e imaturidade seria, no mínimo, muita ingenuidade e imaturidade da nossa parte. Ulisses, para além de um ideal de humano, é um ideal de um homem na Grécia antiga, onde trair e matar muitas pessoas em Troia não era motivo de desonra.

A prova cabal da humanidade recém-adquirida de Ulisses é a sua volta para casa, quando tem de provar sua identidade completamente despojado de quaisquer aspectos físicos ou sociais que pudessem indicar quem ele é. É quando tem de provar a Penélope que é seu marido e a Laertes que é seu filho, e é quando todo o arco narrativo dessa "auto-redescoberta" de Ulisses acaba, quando prova o que batalhou para manter durante toda a jornada: \textbf{memória}.

\nocite{*}
\printbibliography{}

\end{document}